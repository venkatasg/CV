\documentclass[10pt,a4paper]{mycv}
\moderncvstyle{mb}
\moderncvcolor{radish}
\usepackage[hscale=0.65, vscale=0.7]{geometry}
\usepackage{lastpage}

\addbibresource{bibliography.bib}
\nocite{*}

% personal data
\name{VENKATA\space S}{GOVINDARAJAN}
%\phone[mobile]{\href{tel:15125688775}{+1~(512)568-8775}}
\email{GVENKATA1994@GMAIL.COM}
\social[github]{VENKATASG}
\social[linkedin]{VENKATASG1994}
%\social[twitter]{\_venkatasg}
%\homepage{venkatasg.me}
%\address{Dept. of Linguistics}{University of Texas at Austin}{78751}

\begin{document}

\cfoot{\sansc \thepage}
\makecvtitle


\section{education}
% arguments 3 to 6 can be left empty
\cventry{}{PhD Computational Linguistics}
    {University of Texas at Austin}{2019--}{\rmsc cgpa: 3.73/4}{}
\cventry{}{MS Computational Linguistics}
    {University of Rochester}{2017--2019}{\rmsc cgpa: 3.75/4}{}
\cventry{}{Dual Degree(B.Tech \& M.Tech) Biological Engineering}
    {Indian Institute of Technology Madras}{2012--2017}{\rmsc CGPA: 8.68/10}{}
% \cventry{Fall 2015}{Semester exchange}{Czech Technical University Prague}{Prague, Czech Republic}{}{}
% \cventry{May 2012}{Class XII AISSCE}{Bhavans Rajaji Vidyashram, India}{Chennai, India}{Marks:481/500}{}

\section{research interests}

{\large Computational Semantics \& Pragmatics, Natural Language Processing,

\vspace{0.5ex}

Philosophy of Language, Cognitive Science \& Computational Social Science}

%\section{research experience}
%\cventry{}{Research Project - Computational Semantics \& Discourse}
%       {Advice Summarization}{2019-}{Advisor: Prof. Junyi Jessy Li, UT Austin}{}
%        {Our goal is to study the semantics and discourse structures in online
%       advice forums, and build systems capable of hierarchical summarization
%       of advice.}
%\cventry{}{MS Thesis - Computational Semantics}
%        {Decomposing Generalization}{2018-19}
%        {Advisor: Prof. Aaron Steven White, University of Rochester}{}
       % {As part of the Decompositional Semantics Initiative, I
%        developed an annotation protocol and constructed a large-scale
%        corpus of annotations that attempts to moves beyond traditional
%        definitions of genericity by decomposing the referent of
%        arguments and predicates into combinations of simple,
%        real-valued referential properties. I’ve also worked on
%        building computational models that can predict these fine
%        grained distinctions.}
%\cventry{2019}{Research project - Natural Language Processing}
%       {Improving Semantic Parsing Using Statistical WSD}
%       {Univ. of Rochester}{under Ritwick Bose and Prof.James Allen}
%       {\break}
%\cventry{}{M.Tech Thesis - Computational Neuroscience}
%        {Direction Maps in the Whisker Barrel Cortex}
%        {2016-17}{Advisor: Prof. Srinivasa Chakravarthy, IIT Madras}{}
        %{As part of my Masters project, I developed computational models for
%        emergence of position and direction maps in whisker barrel cortex of
%        rat based on self organization principles. I also worked on modelling
%        vasculature in barrel cortex using a lumped parameter model. In the
%        future these two models can be integrated for a neuro-vascular model
%        of the barrel cortex.}
% \cventry{2014}{Summer research intern-Bioinformatics}{TERI(The Energy and Research Institute) University}{Delhi}{}{Under Assistant Professor Pallavi Somvanshi, I worked on analyzing the dbSNP database to identify deleterious SNP’s which lead to polyglutamine disorders. I also used the LAMP(Linux, MySQL, Apache, PHP) stack to build a query server to the dbSNP database. During the course of my internship, I was introduced to various bioinformatic algorithms and tools like BLAST, SIRF, SURF and PolyPhen-2.}

\begingroup
\setlength\bibitemsep{2ex}
\printbibliography[title={\sansc papers}, nottype=unpublished]
\endgroup

\begingroup
\setlength\bibitemsep{2ex}
\printbibliography[title={\sansc talks}, type=unpublished]
\endgroup


\section{teaching}

\subsection{Teaching Assistant}

Analyzing Linguistic Data and Programming for Linguists \hfill Fall 2019

Introduction to Computational Linguistics \hfill Fall 2019

Introduction to Computational Linguistics \hfill Fall 2018

Data Structures and Algorithms for Biology \hfill Fall 2016

\section{skills}

\cvitem{Programming Languages}{Python, Swift, R, MATLAB, LISP, Javascript, C, C++}
\cvitem{Tools \& Frameworks}{pyTorch, SciPy stack, keras, pandas, Docker, \LaTeX, Jupyter, Unix, nltk, SwiftUI, Combine, CoreML}
\cvitem{Languages}{English(fluent), Tamil(fluent), Hindi(intermediate)}

\section{apps}

\cvitem{\href{https://apps.apple.com/us/app/id1531906207}{DeTeXt}}{An open source app for iPhones and iPads that predicts the best LaTeX commands corresponding to hand-drawn symbols using deep neural networks. Built using SwiftUI, Combine, PencilKit and CoreML.}

\section{awards}

Silver medal at International Genetically Engineered Machine (iGEM) 2016.

Indian Biological Engineering Competition (iBEC) grant for INR 1,000,000.

National BIRAC-IdeaThon on Antimicrobial Resistance 2016 Finalist.

Second runner up in 3M-CII Young Innovators Challenge 2015.

\section{coursework}

% \cvitem{Statistical Speech and Language Processing}{During the course I implemented several computational models(in Python) for language processing like HMMs, trigram language models, IBM model 1(for machine translation) and a feed forward neural network for POS tagging.}

% \cvitem{Logical Foundations of AI}{I wrote several programs in LISP for various knowledge representation and reasoning tasks. My course project focused on studying different tests for machine intelligence, understanding their weaknesses, and thus leading towards the goal of a test that might help guide research in AI.}

% \cvitem{Formal Semantics}{As part of a course project, I analysed two competing theories of complement anaphora through an extensive literature review.}
% \cvitem{Principles of Neuroscience}{I proposed a neurobiological model for laughter and humor after an extensive literature review, based on evolutionarily conserved behaviour and the language faculty in humans.}

% \cvitem{Transport Phenomena in Biological Systems}{I worked on a Choose-Focus-Analyse exercise titled `Modeling the spread of Invasive Species'. My goals for this exercise were to design a model which might explain the success of invasive species using simple principles and laws like Fick's laws, law of conservation of mass, and evolution by natural selection.}

% \cvitem{Technical Communications in Biology}{I prepared a complete research proposal for a novel method to image single mRNAs using bacterial coat proteins and FRET.}
\cvitem{}{Language and Power $\bullet$ Syntax $\bullet$ Formal Semantics $\bullet$ Introduction to Pragmatics $\bullet$
          Morphology $\bullet$ Machine Learning $\bullet$
          Statistical Speech and Language Processing $\bullet$
          Logical Foundations of AI $\bullet$ Natural Language Processing
          $\bullet$ Principles of Neuroscience $\bullet$
          Probability,Statistics and Stochastic Processes $\bullet$
          Applied Statistics $\bullet$ Data Structures and Algorithms for
          Biology $\bullet$ Analysis and Interpretation of Biological Data}




% \section{Extracurricular activities}
% \cvitem{Coordinator, Forensics competition and workshop at Shaasta}{I was the co-ordinator of the forensics competition and workshop at Shaastra(IIT Madras's technical festival) 2014. Over 250 people participated in the workshop and competition}
% \cvitem{Volunteer, National Service Scheme}{I volunteered at a school in Kodambakkam, Chennai teaching visually handicapped students.}



\end{document}